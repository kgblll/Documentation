\documentclass{article}
\usepackage{graphicx}
\usepackage[utf8]{inputenc}

\begin{document}

\title{Taller de Dr. Scratch}

\maketitle

\section{Introducción}
Instrucciones para los niños... rodea con un círculo...

\section{La web de Dr. Scratch}
Visita la web de Dr. Scratch: http://drscratch.programamos.es
\begin{enumerate}
  \item ¿Qué te parece la página? ¿Te parece atractiva?
  \item Tras leer la información que aparece en la página principal, ¿para qué crees que sirve Dr. Scratch?
\end{enumerate}

\section{Análisis de un proyecto}
Analiza uno de tus proyectos Scratch con Dr. Scratch.
\begin{enumerate}
  \item ¿Te ha resultado sencillo realizar el análisis?
  \item ¿Qué puntuación has obtenido?
  \item Según Dr. Scratch, ¿a qué nivel corresponde esa puntuación?
  \item ¿Cómo te has sentido al ver los resultados?
  \item ¿Por qué?
\end{enumerate}

\section{Ideas para mejorar un proyecto}
Desde la página de resultados, tras analizar un proyecto, pincha en alguno de los enlaces para recibir información que pueda ayudarte a mejorar tu proyecto.
\begin{enumerate}
  \item Escribe el nombre de la página en la que has pinchado.
  \item ¿Te resulta comprensible la información que se muestra en esta nueva página?
  \item Tras leer la información, ¿tienes ganas de mejorar tu proyecto probando algo nuevo en Scratch?
\end{enumerate}

\section{Mejorando un proyecto}
Utilizando la información que se ha mostrado en la página de ayuda seleccionada, intenta mejorar tu proyecto incorporando algo nuevo. 
\begin{enumerate}
  \item ¿Es suficiente la información mostrada para comprender cómo realizar la mejora?
  \item Tras realizar las mejoras siguiendo las instrucciones de la página de ayuda, analiza de nuevo tu proyecto con Dr. Scratch. ¿Has obtenido una puntuación mayor?
\end{enumerate}

\section{Comentarios finales}
Comenta todo lo que quieras sobre Dr. Scratch: cosas que te gustan, cosas que no, ideas que podríamos incorporar...

\end{document}