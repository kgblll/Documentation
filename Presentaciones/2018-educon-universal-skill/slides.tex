%
% $Id: $
%
%
% Compilar a .pdf con LaTeX (pdflatex)
% Es necesario instalar Beamer (paquete latex-beamer en Debian)
%

%
% Gr�ficos:
% Los gr�ficos pueden suministrarse en PNG, JPG, TIF, PDF, MPS
% Los EPS deben convertirse a PDF (usar epstopdf)
%

\documentclass{beamer}
\usetheme{Warsaw}
%\usebackgroundtemplate{\includegraphics[width=\paperwidth]{format/libresoft-bg.png}}
%\usepackage[spanish]{babel}
\usepackage[latin1]{inputenc}
\usepackage{graphics}
\usepackage{amssymb} % Simbolos matematicos
\usepackage{url}
\usepackage{multirow}
\usepackage{subfigure}

\addtobeamertemplate{navigation symbols}{}{%
    \usebeamerfont{footline}%
    \usebeamercolor[fg]{footline}%
    \hspace{1em}%
    \insertframenumber/\inserttotalframenumber
}

%\definecolor{libresoftgreen}{RGB}{162,190,43}
%\definecolor{libresoftblue}{RGB}{0,98,143}

%\setbeamercolor{titlelike}{bg=libresoftgreen}

%% Metadatos del PDF.
\hypersetup{
  pdftitle={On Computational Thinking as a Universal Skill: A review of the latest research on this ability},
  pdfauthor={Gregorio Robles},
  pdfcreator={Kindergarten and Beyond - Lifelong Learning Research Group (KGB-L3) \\ Universidad Rey Juan Carlos},
  pdfproducer=PDFLaTeX,
  pdfsubject={Computational Thinking Research @ URJC},
}
%%

\begin{document}

\title{On Computational Thinking as a Universal Skill}
\subtitle{A review of the latest research on this ability}
\institute{INTEF, Univ. Rey Juan Carlos, Univ. Nacional de Educaci�n a Distancia \\ (Madrid, Spain)}
\author[Moreno-Le�n et al. // jmorenol@gmail.com]{Jes�s Moreno-Le�n, Gregorio Robles, Marcos Rom�n-Gonz�lez}
\date{EDUCON 2018, Santa Cruz de Tenerife, April 19\textsuperscript{th} 2018}


\AtBeginSection{\frame{\sectionpage}}

\frame{
\maketitle
\begin{center}
\includegraphics[width=1.4cm]{format/intef}
\hspace{0.5cm}
\includegraphics[width=3.2cm]{format/urjc}
\hspace{0.5cm}
\includegraphics[width=1.3cm]{format/uned.jpg}
\hspace{0.5cm}
\includegraphics[width=3cm]{format/emadrid.png}
\end{center}
}


% Si el titulo o el autor se quieren acortar para los pies de p�gina
% se pueden redefinir aqu�:
%\title{Titulo corto}
%\author{Autores abreviado}

%% LICENCIA DE REDISTRIBUCION DE LAS TRANSPAS
\frame{
~
\vspace{3cm}

\begin{flushright}
\includegraphics[width=2.2cm]{figs/by-sa}
\vspace{0.3cm} \\
{\tiny
(cc) 2018 Jes�s Moreno-Le�n, Gregorio Robles, Marcos Rom�n\\
  Some rights reserved. This work licensed under Creative Commons \\
  Attribution-ShareAlike License. To view a copy of full license, see \\
  http://creativecommons.org/licenses/by-sa/3.0/ or write to \\
  Crea  tive Commons, 559 Nathan Abbott Way, Stanford, \\
  California 94305, USA. \\
\ \\
Some of the figures have been taken from the Internet \\
Source, and author and license if known, is specified. \\
For those images, \emph{fair use} applies.
}
\end{flushright}
}
%%


%--------------------------------------------------------

\begin{frame}
\frametitle{Authors}

\begin{center}
	\begin{figure}[t!]
		\includegraphics[width=2.5cm,height=2.8cm]{figs/jemole}
		\hspace{0.6cm}
		\includegraphics[width=2.5cm,height=2.8cm]{figs/gregoriorobles}    
    	\hspace{0.9cm}
		\includegraphics[width=2.5cm,height=2.8cm]{figs/marcos}
%        \vspace{0.4cm}
		\\ \hspace{0.3cm} Jes\'us Moreno-Le\'on \hspace{0.2cm} Gregorio Robles \hspace{0.1cm}  Marcos Rom\'an-Gonz\'alez
		\vspace{0.1cm}
		\\ INTEF \hspace{2cm} URJC \hspace{2.3cm} UNED
	\end{figure}
\end{center}


\end{frame}


%%--------------------------------------------------------

\begin{frame}
\frametitle{History of learning \emph{with} computers}
\begin{center}
        \begin{figure}[t!]
                \includegraphics[width=11cm]{figs/history.png}
                \caption{You are here!}
        \end{figure}
        \tiny Source: \textit{REaCT EU project proposal}
\end{center}

\end{frame}

\usebackgroundtemplate{}


%--------------------------------------------------------

\begin{frame}
\frametitle{Great Principles of Computing}

  \begin{columns}[T]
    \begin{column}{0.5\textwidth}
     \begin{block}{The Book}
\begin{figure}[t!]
\begin{center}
\includegraphics[width=4.4cm,height=5.6cm]{figs/gpc.jpeg}
\end{center}
\label{fig:gpc}
\end{figure}
     \end{block}
    \end{column}
    \begin{column}{0.5\textwidth}
 
     \begin{block}{Seven Principles}
\begin{enumerate}
  \item Information
  \item Machines
  \item Programming
  \item Computation
  \item Memory
  \item Parallelism
  \item Queuing
  \item and Design
\end{enumerate}
     \end{block}

    \end{column}
  \end{columns}


\end{frame}

\usebackgroundtemplate{}


%--------------------------------------------------------

\begin{frame}
\frametitle{We still don't know what Computational Thinking is...}

\begin{figure}[htb!]
  \centering
  \includegraphics[width=.95\columnwidth]{figs/CT-cloudword.jpg}
  \caption{The term CT is ambiguous and vague... but it is not computing!}
  \label{fig:CT-cloud}
\end{figure}


\end{frame}

\usebackgroundtemplate{}



%--------------------------------------------------------

\begin{frame}
\frametitle{Is CT a new skill? (I)}

\begin{figure}[htb!]
  \centering
  \includegraphics[width=.63\columnwidth]{figs/CT1.png}
  \caption{Computational thinking: only 27\% of its variance can be explained by the four primary mental abilities (i.e., reasoning ability, visual ability, verbal ability, and numerical ability).}
  \label{fig:CT1}
\end{figure}


\end{frame}

\usebackgroundtemplate{}

%--------------------------------------------------------

\begin{frame}
\frametitle{Is CT a new skill? (and II)}

\begin{figure}[htb!]
  \centering
  \includegraphics[width=.66\columnwidth]{figs/CT2.png}
  \caption{Computational thinking: only 51\% of its variance can be explained by the four primary mental abilities and non-cognitive personality factors.}
  \label{fig:CT2}
\end{figure}


\end{frame}

\usebackgroundtemplate{}


%--------------------------------------------------------

\begin{frame}
\frametitle{How to develop CT?}


  \begin{columns}[T]
    \begin{column}{0.5\textwidth}
     \begin{block}{Unplugged activities}
\begin{figure}[t!]
\begin{center}
\includegraphics[width=4.4cm,height=3.6cm]{figs/unplugged.png}
\end{center}
\label{fig:unplugged}
\end{figure}
\begin{center}
Unplugged activities \\ (no use of digital devices)
\end{center}
     \end{block}
    \end{column}
    \begin{column}{0.5\textwidth}
 
     \begin{block}{Programming}
\begin{figure}[t!]
\begin{center}
\includegraphics[width=4.4cm,height=4.2cm]{figs/programming.png}
\end{center}
\label{fig:programming}
\end{figure}
\begin{center}
Several ways (arrow-based, block-based, textual, connected to physical world)
\end{center}
     \end{block}
    \end{column}
  \end{columns}


\end{frame}

\usebackgroundtemplate{}




%%--------------------------------------------------------

\begin{frame}
\frametitle{What should be the approach?}

\vspace{-0.4cm}
\begin{center}
        \begin{figure}[t!]
                \includegraphics[width=10cm]{figs/facetoface.jpg}
                \caption{Code to learn vs. learn to code}
        \end{figure}
\end{center}

\end{frame}

\usebackgroundtemplate{}




%%--------------------------------------------------------

\begin{frame}
\frametitle{We argue for ``Code to learn''}

\vspace{-0.4cm}
\begin{center}
        \begin{figure}[t!]
                \includegraphics[width=10cm]{figs/resnick.png}
                \caption{Mitch Resnick (MIT Media Lab)}
        \end{figure}
\end{center}

\end{frame}

\usebackgroundtemplate{}




%%--------------------------------------------------------

\usebackgroundtemplate{}
%background: 

\begin{frame}
\frametitle{Code to learn: Maths}

\begin{figure}[h!]
  \centering
	\includegraphics[width=.80\textwidth]{figs/maths.png}
  \caption{Coraz�n de Mar�a school. Comparing the improvement between pre- and post-tests of control and experimental groups.}
  \label{fig:corazon}
\end{figure}

\end{frame}

\usebackgroundtemplate{}


%%--------------------------------------------------------

\usebackgroundtemplate{}
%background: 
    
\begin{frame}
\frametitle{Code to learn: Social sciences}


\begin{figure}[h!]
  \centering
	\includegraphics[width=.80\textwidth]{figs/socialstudies.png}
  \caption{La Jota school. Comparing the improvement between pre- and post-tests of control and experimental groups.}
  \label{fig:jota}
\end{figure}

\end{frame}

\usebackgroundtemplate{}

%%--------------------------------------------------------

\usebackgroundtemplate{}
%background: 

\begin{frame}
\frametitle{Code to learn: Arts}


\begin{figure}[h!]
  \centering
	\includegraphics[width=.80\textwidth]{figs/languagearts.png}
  \caption{La Inmaculada school. Comparing the improvement between pre- and post-tests of control and experimental groups.}
  \label{fig:inma}
\end{figure}



\end{frame}

\usebackgroundtemplate{}

%--------------------------------------------------------
\begin{frame}
\frametitle{Code to learn: Social learning}

\begin{figure}[h!]
\begin{subfigure}
  \centering
    \includegraphics[width=.48\textwidth]{figs/blocks_social2.png}
\end{subfigure} 
\begin{subfigure}
  \centering
    \includegraphics[width=.48\textwidth]{figs/types_social2.png}
  \caption{(l) Relationship of level of sociability with improvement in depth for each time group. (r) Relationship of level of sociability on improvement in breadth for each time group.}
  \label{fig:blocks_social2}
\end{subfigure}
\end{figure} 

\end{frame}

\usebackgroundtemplate{}


%%--------------------------------------------------------

\usebackgroundtemplate{\includegraphics[height=8.8cm]{figs/research.jpg}}
%background: 

\begin{frame}
\frametitle{Research in Progress: Art with 3-year old kids}



\begin{figure}[htb!]
  \centering
  \includegraphics[width=.75\columnwidth]{figs/beebot.jpg}
  \caption{A five year old student coding a programmable robot to develop arts skills and learn about the painting \emph{Bathers at Asni�res}.}
  \label{fig:beebot}
\end{figure}


\vspace{0.2cm}
\hfill{\Tiny Background picture: Pixabay - Public domain}

\end{frame}

\usebackgroundtemplate{}



%--------------------------------------------------------

\begin{frame}
\frametitle{The quest for assessment}

  \begin{columns}[T]
    \begin{column}{0.5\textwidth}
     \begin{block}{CT-test}
\begin{figure}[t!]
\begin{center}
\includegraphics[width=4.8cm,height=3.6cm]{figs/comecocos.png}
\end{center}
\label{fig:cttest}
\end{figure}
\begin{center}
diagnostic-summative paper-and-pen test
\end{center}
     \end{block}
    \end{column}
    \begin{column}{0.5\textwidth}
 
     \begin{block}{Dr. Scratch}
\begin{figure}[t!]
\begin{center}
\includegraphics[width=4.8cm,height=3.6cm]{figs/drscratch2.png}
\end{center}
\label{fig:drscratch}
\end{figure}
\begin{center}
formative assessment tool
\end{center}
     \end{block}

    \end{column}
  \end{columns}
\begin{center}
Other methods exist, such as Bebras \\ (which assesses skill transference)
\end{center}


\end{frame}

\usebackgroundtemplate{}




%%--------------------------------------------------------
\usebackgroundtemplate{\includegraphics[width=13cm]{figs/take-away.jpg}}
%% background: http://flamingcow.co.uk/wp-content/uploads/2015/02/takeaway-940x283.jpg

\begin{frame}
\frametitle{Takeaways}

  \begin{enumerate}
    \item Computational Thinking is a \emph{different} skill
    \item We argue that focus should not be on programming, but on (transversal) skills
    \item We need more research on the impact of CT on other subjects, including tools that facilitate this task
    \item Social skills are also to be taken into account
    \item We need better assessment tools
  \end{enumerate}

\vspace{0.85cm}
\hfill{\Tiny Background picture: flamingcow.co.uk}
%
\end{frame}

\usebackgroundtemplate{}

%--------------------------------------------------------
\usebackgroundtemplate{\includegraphics[width=13cm]{figs/books.jpg}}
% http://www.aspa-usa.org/wp-content/uploads/2015/06/books.jpg
\begin{frame}
\frametitle{Learn more}
\vspace{-0.65cm}
\begin{center}
\footnotesize
\begin{columns}[T]
    \begin{column}{1\textwidth}
     \begin{block}{Some references}
       \begin{enumerate}
	 \item J. Moreno-Le\'on, G. Robles, M. \& Roman-Gonz\'alez. ``Dr. Scratch: Automatic Analysis of Scratch Projects to Assess and Foster Computational Thinking''. \textit{RED. Revista de Educaci�n a Distancia, 15}(46). 2015
	 \item J. Moreno-Le�n, G. Robles, and M. Rom�n-Gonz�lez. ``Examining the Relationship between Socialization and Improved Software Development Skills in the Scratch Code Learning Environment.'' Journal of Universal Computer Science, 22(12), pp. 1533-1557. 2016
	 \item M. Rom�n-Gonz�lez, J.C. P�rez-Gonz�lez, J. Moreno-Le�n, G. \& Robles. ``Extending the nomological network of computational thinking with non-cognitive factors.'' Computers in Human Behavior. 2018
       \end{enumerate}
    \end{block}
    \end{column}
  \end{columns}
\end{center}
\end{frame}

\usebackgroundtemplate{}

\frame{
\maketitle
\begin{center}
\includegraphics[width=1.4cm]{format/intef}
\hspace{0.5cm}
\includegraphics[width=3.2cm]{format/urjc}
\hspace{0.5cm}
\includegraphics[width=1.3cm]{format/uned.jpg}
\hspace{0.5cm}
\includegraphics[width=3cm]{format/emadrid.png}
\end{center}
}


\end{document}
