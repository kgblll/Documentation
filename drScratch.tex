%Dr.Scratch's Documentation
%Author: Mari Luz Aguado Jiménez

%PREAMBLE
\documentclass[a4paper, 12pt]{book}

\usepackage{lmodern}
\usepackage{fontenc}
\usepackage[a4paper, left=2.5cm, right=2.5cm, top=3cm, bottom=3cm]{geometry}
\usepackage{mathtools}
\usepackage{color}
\usepackage[latin1]{inputenc}
\usepackage{url}
\usepackage{graphicx}
\usepackage[nottoc, notlot, notlof, notindex]{tocbibind} 


\title{Dr.Scratch Documentation}
\author{Mari Luz Aguado Jiménez}
\renewcommand{\baselinestretch}{1.5}  %% Interlineado
\date{11 March 2015}
%//END PREAMBLE

%BODY OF THE DOCUMENT
\begin{document}

%\renewcommand{\refname}{Bibliografía}  %% Renombrando
\renewcommand{\appendixname}{Apéndice}

%%%%%%%%%%%%%%%%%%%%%%%%%%%%%%%%%%%%%%%%%%%
				% PORTADA %
%%%%%%%%%%%%%%%%%%%%%%%%%%%%%%%%%%%%%%%%%%%
\begin{titlepage}
\begin{center}

\begin{tabular}[c]{c c}
\vspace{2cm}
\end{tabular}



\begin{tabular}[c]{c c}
\Huge
DR.SCRATCH'S DOCUMENTATION
\end{tabular}

\vspace{6cm}

\includegraphics[scale=0.7]{img/logoURJC.jpg}
\begin{tabular}[b]{l}
\includegraphics[scale=0.7]{img/logoProgramamos.png} 
\end{tabular}
\\
\includegraphics[scale=0.5]{img/logoLibresoft.jpg}
\begin{tabular}[b]{l}
\includegraphics[scale=0.5]{img/logoFECYT.png} 
\end{tabular}


\vspace{2.5cm}

\vspace{4cm}

\large
2014/2015
\end{center}
\end{titlepage}

\newpage
\mbox{}
\thispagestyle{empty} % don't show anything in this page

%%%%%%%%%%%%%%%%%%%%%%%%%%%%%%%%%%%%%%%
			% INDEX %
%%%%%%%%%%%%%%%%%%%%%%%%%%%%%%%%%%%%%%%
\chapter*{}
\pagenumbering{Roman} % Roman numerals
\begin{flushright}
\textit{Our passion is our strength.}
\end{flushright}


\tableofcontents  % Index of contents
\cleardoublepage
\listoffigures % Index of figures
%\cleardoublepage
%\listoftables % Index of tables

%%%%%%%%%%%%%%%%%%%%%%%%%%%%%%%%%%%%%%%%%%%%%
		      % INTRODUCTION %
%%%%%%%%%%%%%%%%%%%%%%%%%%%%%%%%%%%%%%%%%%%%%
\cleardoublepage
\pagenumbering{arabic} % Arabic numerals
\chapter{Getting started with Dr.Scratch}
\label{sec:beginning} % to refer in the text ~\ref{sec:introduction}
\section{The beginning} 
\setlength{\parskip}{8mm}

\textsl{'
Dr.Scratch is a web application to analyze your own (or their students) Scratch
projects. Feedback is given in a variety of areas Computational Thinking. That
way, you or your students can be able to learn and improve your programming
skills.'}

To begin with, we want to show the evolution of Dr.Scratch and how it works
without going into technical details. The reason of this is we would like have
a view of our errors at the beginning and how we did solve thank to the
feedback of our users.

Dr.Scratch begins as a simple idea, correct bad habits that many students can
acquire programming. Is being introduced the Scratch environment as a tool to
teach programming skills or develop computational thinking is increasingly
common in all levels of education and Dr.Scratch tries to help in this mission.
 
At first, we thought about the Scratch projects which you can download from
Scratch web how the Figure~\ref{figure:download} shows. It allows you to have your project in your own computer and upload it to
Dr.Scratch to be analyzing. 
 
\begin{figure}
    \centering
    \includegraphics[scale=0.7]{img/download.png}
    \caption{Downloading a project from Scratch to your computer.}
    \label{figure:download}
\end{figure}

The first appearance that we designed was very simple to use this option and it 
helped us to implement the functionality of upload of file and analysis. You
can take a look at the Figure~\ref{figure:first-appearance} shows.

 \begin{figure}
   	\centering
   	\includegraphics[scale=0.7]{img/first-appearance.png}
   	\caption{First Appearance of Dr.Scratch.}
   	\label{figure:first-appearance}
\end{figure}
 
You could choose a file (previously downloaded from Scratch), click
`` Send '' to analyze and find code smells with the information 
provided on the dashboard that Dr.Scratch showed  after a few seconds.
   

Also you had got extra information:
\setlength{\parskip}{2mm}
\begin{itemize}

   	\item A demo video explaining how to use Dr.Scratch to improve your 
   	programming skills.
   	\item A contact section where you could find our e-mail and twitter to write
   	us if you wanted.
	\item The new features that we wanted to develop in that moment.
	\item The posibility to sign in on the top.
\end{itemize}
\setlength{\parskip}{8mm}
When you clicked on ``Send'' Dr.Scratch uploaded your project to our server
without to save any additional information. After that, we used and
still use some plug-ins (which will be described later) to analyze the 
project and showed the dashboard you can see in the 
Figure~\ref{figure:first-dashboard}.

The dashboard gives you information in detail about several aspects which are
related to Computational Thinking like: 
\setlength{\parskip}{0mm}
\begin{itemize}
  	\setlength{\parskip}{0mm}
   	\item Logic
   	\item Abstraction
	\item Parallelization
	\item Synchronization
	\item Flow Control
	\item User Interactivity
	\item Data Representation
\end{itemize}

\begin{figure}
   	\centering
   	\includegraphics[scale=0.3]{img/first-dashboard.png}
   	\caption{First Dashboard of Dr.Scratch.}
   	\label{figure:first-dashboard}
\end{figure}

In accordance with these aspects, your project will have a score that is showed
in the top with the title ``CT Score'' and gives you a idea about your level.

The other information provided tells about those habits which some programmers
can acquire when starts in this world. We have started with four of these bad
habits:
\setlength{\parskip}{0mm}
\begin{enumerate}
  	\setlength{\parskip}{0mm}
   	\item Duplicated scripts
   	\item Sprite naming
	\item Dead code
	\item Sprite attributes initialization
\end{enumerate}


This appearance was in production since November of 2014 until March of 
2015. Although it was very simple, had high acceptance. We reckon this was 
the reason of his successful. But we have a long way to go, in fact, we are
including new features and appearances trying to get close to the final product.
These new features and appearances will be described in the following section
explaining why we thought it was a good idea add it.


\section{Always improving}



\end{document}
